\documentclass[12pt]{article}
\usepackage[T1]{fontenc}
\usepackage[latin9]{inputenc}
\usepackage{textcomp}
\usepackage{amstext}
\usepackage{graphicx}
\usepackage{amssymb}
\makeatletter
\providecommand{\tabularnewline}{\\}

\usepackage{fourier-orns}
\usepackage[colorlinks,linkcolor=blue]{hyperref}
\usepackage{footmisc}
\usepackage{subfigure}
\usepackage{float}

\topmargin=-0.5in
\oddsidemargin = 0in
\parindent=0.0cm
\parskip=0.3cm
\textwidth=6.5in
\textheight= 10in

\sloppy
\newtheorem{theorem}{Theorem}[section]
\newtheorem{lemma}{Lemma}[section]
\newtheorem{corollary}{Corollary}[section]
\newtheorem{definition}{Definition}[section]


\begin{document}
\pagestyle{empty}

{\bf To what degree can tools and automation help us to recover from system faults?}

There are a variety of reasons that can make it difficult for the human to control the system while using it. Developing an autonomic system that can self-control self-recovery has increased in recent years. However, the software may fail in different ways it might be malfunctioning to run the self-healing function. \\[10px]
Google has designed a large scale distributed system tracing infrastructure Dapper. It is such an infrastructure that can generate a lot of useful information without application involvement. This is designed whereas the system is continuously developed by different teams meaning "anyone is an expert in the internals of all of them"\cite{ct4}.\\[10px]
The paper written by Amjad A. Hudaib has addressed that Software Fault Detection and Recovery (SFDR) method detects the cases if a fault occurs with software components such as component deletion, replacement or modification \cite{ct2}. It can automatically diagnose monitor and recover the fault component in the target software. The application of this method has been implemented to demonstrate that it is better than other industry approaches such as Microsoft windows restore ASURE in the case of recover error result from deleting software components and fault recovery.\\[10px]
Research has shown that there are many ways that we can recover the system fault even in the very low level, such as using machine learning in the assembly levels. In this paper written by L. Seabra Lopes, they addressed that through the use of machine learning techniques, the supervision architecture will be given capabilities for improving its performance in the unforeseen event \cite{ct1}. Through training the machine learning model using the raw sensor. An error classification algorithm is subsequently generated by induction algorithm SKIL. This approach has been tested on an industry system on a robot called SCARA.\\[10px]
Furthermore, a system can be a self-fault tolerant system itself by using big data \cite{ct3}. By using the technology related to the Internet of Things such as wireless sensor and cloud computing. We can make the fault diagnosis being data driven instead of knowledge-driven which means the full automation testing can be achieved driven by big data. In this paper, it has been stated that by using deep learning network such as hierarchical diagnosis network or deep belief learning (DBN) to extra the feature from the data can help us to gain the insight view into the reason of the fault. It has a range of industrial applications such as electrocardiogram fault diagnosis system. 
{\footnotesize
\begin{thebibliography}{9}
  \bibitem{ct1}
    L. Seabra Lopes and L. M. Camarinha-Matos, \textbf{A machine learning approach to error detection and recovery in assembly}
    \url{https://ieeexplore.ieee.org/document/525884}
    \textit{Date Accessed: 15 February 2019}

    \bibitem{ct2}
    A. Hudaib and H. Fakhouri, \textbf{An Automated Approach for Software Fault Detection and Recovery}
    \url{https://www.scirp.org/journal/PaperInformation.aspx?PaperID=69412}
    \textit{Date Accessed: 15 February 2019}

    \bibitem{ct3}
    Y. Xu, Y. Sun, J. Wan, X. Liu and Z. Song, \textbf{Industrial Big Data for Fault Diagnosis: Taxonomy, Review, and Applications}
    \url{https://ieeexplore.ieee.org/document/7990488}
    \textit{Date Accessed: 16 February 2019}

    \bibitem{ct4}
    Sigelman, \textbf{Dapper, A Large Scale Distributed Systems Tracing Infrastructure}
    \url{https://blog.acolyer.org/2015/10/06/dapper-a-large-scale-distributed-systems-tracing-infrastructure/}
    \textit{Date Accessed: 16 February 2019}
\end{thebibliography}
}

\end{document}